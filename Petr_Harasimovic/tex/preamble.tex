\documentclass{beamer}
%\documentclass[handout]{beamer}


\mode<presentation>
{
  \usetheme{CambridgeUS}
  % or ...
% good themes are: Warsaw, Malmoe, JuanLesPins, Montpellier, Pittsburgh, Boadilla, luebeck, Hannover, montpellier, Pittsburgh(?),

\usecolortheme{seahorse}
\usecolortheme{lily}

 % \setbeamercovered{transparent}
  % or whatever (possibly just delete it)
}

\usepackage[english]{babel}

\usepackage{wasysym}


\usepackage{listings}
\usepackage{color}
 
\definecolor{codegreen}{rgb}{0,0.6,0}
\definecolor{codegray}{rgb}{0.5,0.5,0.5}
\definecolor{codepurple}{rgb}{0.58,0,0.82}
\definecolor{backcolour}{rgb}{0.95,0.95,0.92}
 
\lstdefinestyle{mystyle}{
    backgroundcolor=\color{backcolour},   
    commentstyle=\color{codegreen},
    keywordstyle=\color{magenta},
    numberstyle=\tiny\color{codegray},
    stringstyle=\color{codepurple},
    basicstyle=\footnotesize,
    breakatwhitespace=false,         
    breaklines=true,                 
    captionpos=b,                    
    keepspaces=true,                 
    numbers=left,                    
    numbersep=5pt,                  
    showspaces=false,                
    showstringspaces=false,
    showtabs=false,                  
    tabsize=2
}
 
\lstset{style=mystyle,
        language=fortran}


\title[Bath bug hunt] % (optional, use only with long paper titles)
{Fortran Dynamic Dispatching Bug}

%\subtitle
%{Include Only If Paper Has a Subtitle}

\author[Petr Harasimovi\v c] % (optional, use only with lots of authors)
{Petr Harasimovi\v c\\{\footnotesize UoB} } % \inst{1} } %\and S.~Another\inst{2}}
% - Give the names in the same order as they appear in the paper.
% - Use the \inst{?} command only if the authors have different
%   affiliation.


\date[Bath June 2018] % (optional, should be abbreviation of conference name)
%{Feb 3, 2014}
{\today}
% - Either use conference name or its abbreviation.
% - Not really informative to the audience, more for people (including
%   yourself) who are reading the slides online


\institute[UoB] % (optional, but mostly needed)
%{
%  \inst{1}%
%  Center for Economic Research and Graduate Education -- Economic Institut}
  %\and
  %\inst{2}%
  %Department of Theoretical Philosophy\\
  %University of Elsewhere}
% - Use the \inst command only if there are several affiliations.
% - Keep it simple, no one is interested in your street address.


%\subject{Theoretical Computer Science}
% This is only inserted into the PDF information catalog. Can be left
% out.



% If you have a file called "university-logo-filename.xxx", where xxx
% is a graphic format that can be processed by latex or pdflatex,
% resp., then you can add a logo as follows:

% \pgfdeclareimage[height=0.5cm]{university-logo}{uob_logo_full_colour}
% \logo{\pgfuseimage{university-logo}}


\AtBeginSection[]
{
  \begin{frame}<beamer>
    \frametitle{Outline}
    \tableofcontents[currentsection]
  \end{frame}
}


% Delete this, if you do not want the table of contents to pop up at
% the beginning of each subsection:
\AtBeginSubsection[]
{
  \begin{frame}<beamer>
    \frametitle{Outline}
    \tableofcontents[currentsection,currentsubsection]
  \end{frame}
}


% If you wish to uncover everything in a step-wise fashion, uncomment
% the following command:

%\beamerdefaultoverlayspecification{<+->}


