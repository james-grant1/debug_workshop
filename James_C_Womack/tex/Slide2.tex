\documentclass{beamer}
\mode<presentation>
\beamertemplatenavigationsymbolsempty 

\title{Debugging Numerical Software}
\author{James C. Womack}

\setbeamertemplate{footline}[text line]{%
  \strut 
  \insertauthor 
  \hfill
  \inserttitle
}

\begin{document}
\begin{frame}
  \frametitle{Bug solution}
  \begin{description}
    \item[Current status] Resolved, but unsure why!
    % - Using an explicit SAVE attribute solves the problem, but according to
    %   the standard, this should not be necessary (I think).
    %   - GFortran does not suffer from this issue at all, apparently.
    % - I think the difficulty in preparing a MWE is an interesting topic to 
    %   discuss.
    %   - It is unclear whether this is a compiler bug, or a consequence of 
    %     some subtle error elsewhere in the code.
    %   - Without access to the ONETEP codebase, it is difficult to really
    %     investigate this deeper.
  \end{description}
  {\footnotesize
  \begin{itemize}
    \item While investigating this bug, I discovered some useful information about GFortran and Intel Fortran...
    \item Compiling with OpenMP flags causes compilers to silently apply flags which change how variables are allocated:
    \begin{description}
      \item[GFortran] \texttt{-fopenmp} implies \texttt{-frecursive}, which forces all local arrays to be allocated on the stack.
      % See https://gcc.gnu.org/onlinedocs/gfortran/OpenMP.html
      \item[Intel Fortran] \texttt{-qopenmp} implies \texttt{-automatic}, which causes all local, non-SAVEd variables to be allocated on the run-time stack.
      % See https://software.intel.com/en-us/download/intel-fortran-compiler-170-developer-guide-and-reference (p.315)
      %
      % - I originally thought that this was the cause of the issue, i.e.
      %   that -qopenmp causing non-SAVEd variables to be allocated to the 
      %   run-time stack explained why the implicitly SAVEd local variable was
      %   losing its value.
      % - But, the fact that the issue does not appear for GFortran and a MWE
      %   cannot be prepared simply by combining an implicit SAVE attribute with
      %   OpenMP suggests something more subtle.
      % - This should not affect standards-compliant, properly written code, 
      %   but may cause issues where you code was accidentally relying on 
      %   non-SAVEd variables being statically allocated.
      %   --> A useful piece of information for debugging code where the 
      %       behaviour changes unexpectedly with/without OpenMP.
    \end{description}
    \item Possible causes:
    \begin{itemize} 
      \item Compiler bug/non-conforming compiler
      \item Subtle memory corruption issue in code
      \item Something completely different...?
    \end{itemize}
  \end{itemize}
  }

\end{frame}
\end{document}
